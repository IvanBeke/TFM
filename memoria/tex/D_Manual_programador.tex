\apendice{Documentación técnica de programación}

\section{Introducción}
En este apéndice se presenta todo lo que tiene que conocer un desarrollador para poder continuar con el desarrollo de la aplicación. Se describe la estructura de directorios del proyecto, cómo instalar la aplicación, etc.

El proyecto tiene dos partes diferenciadas en cuanto a desarrollo. La primera es una colección de \textit{notebooks} en la que se realizar pruebas con la API y algoritmos. La segunda es una aplicación web Django estándar.

\section{Estructura de directorios}

\dirtree{%
.1 / Directorio raíz.
.2 memoria/ - Documentación del proyecto.
	.3 img/ - Imágenes de la memoria.
	.3 tex/ - Secciones de la memoria.
	.3 memoria.tex - Código fuente de la memoria.
	.3 memoria.pdf - Memoria del proyecto.
	.3 bibliografia.bib - Fuentes bibliográficas.
.2 scripts and notebooks/ - Colección de notebooks y scripts.
	.3 outputs/ - Directorio donde se almacenan las salidas de los notebooks.
	.3 0\_extract\_players.ipynb - Notebook para extraer jugadores.
	.3 1\_extract\_matches.ipynb - Notebook para extraer partidas.
	.3 2\_download\_matches\_details.ipynb - Notebook para obtener detalles de cada partida.
	.3 3\_algorithms.ipynb - Notebook para probar algoritmos.
	.3 utils.py - Colección de funciones comunes para los notebooks.
	.3 api\_key.txt - Fichero que contiene la clave de la API.
	.3 requirements.txt - Fichero que lista las dependencias.
.2 web/ - Directorio del proyecto en Django.
	.3 betterbuilds/ - App de Django que contiene la web.
		.4 migrations/.
		.4 static/.
		.4 templates/.
		.4 \_\_init.py\_\_.
		.4 admin.py.
		.4 apps.py.
		.4 models.py.
		.4 tests.py.
		.4 urls.py.
		.4 views.py.
	.3 etl/ - App de Django que contiene el proceso ETL.
		.4 management/.
			.5 commands/.
		.4 migrations/.
		.4 \_\_init.py\_\_.
		.4 admin.py.
		.4 apps.py.
		.4 models.py.
		.4 tests.py.
		.4 views.py.
	.3 web/ - Directorio de configuración del proyecto Django.
		.4 \_\_init.py\_\_.
		.4 asgi.py.
		.4 settings.py.
		.4 urls.py.
		.4 utils.py.
		.4 wsgi.py.
	.3 .env - Fichero con variables de entorno.
	.3 manage.py - Script para interactuar con el proyecto.
	.3 requirements.txt - Fichero que lista las dependencias.
}

\section{Manual del programador}
\subsection{Notebooks}
Como los \textit{notebooks} han sido usados para realizar pruebas y probar la \textit{pipeline}, se recomienda probar en los mismos cualquier modificación a realizar en la recopilación y transformación de los datos.

También es recomendable usarlos para explorar los parámetros en las diversas peticiones a la API y sus respuestas.

\subsection{Aplicación web}
Al ser una una aplicación estándar de \textit{Django}, se necesita tener conocimiento sobre el \textit{framework}. Para cualquier duda consultar en su \hrefFootnote{https://docs.djangoproject.com/en/3.2/}{documentación}.

En este apartado se explican de forma básica los conceptos del \textit{framework} y cómo se han aplicado en este desarrollo.

Los proyectos escritos en \textit{Django} se organizan en los que llaman \textit{apps}. Cada \textit{app} representa cada módulo o servicio del proyecto, se componen de vistas y modelos propios e independientes del resto de \textit{apps}, aunque pueden integrarse para el correcto funcionamiento del proyecto. El producto desarrollado se compone de dos \textit{apps}, \textbf{betterbuilds} y \textbf{etl}.

La \textit{app} betterbuils contiene las vistas y modelos que se encargan de pedir y mostrar los datos necesarios en el navegador. Esto incluye la pantalla de inicio, el listado de campeones y la ficha del campeón. Sería la aplicación web comúnmente hablando.

Dentro de la \textit{app} etl se encuentran las distintas fases de la \textit{pipeline} codificadas como comandos de gestión del proyecto. Los comandos de gestión se ejecutan desde el terminal y funcionan como si se tratase de la ejecución de un \textit{script}. Para cada fase de la \textit{pipeline} existe un comando que la realiza de principio a fin. La relación entre los distintos comandos y la fase es la siguiente:
\begin{description}
	\item[Extracción de jugadores:] \code{import\_summoners.py}
	\item[Extracción de partidas:] \code{import\_matches.py}
	\item[Añadir posición:] \code{add\_position.py}
	\item[Generar transacciones:] \code{generate\_transactions.py}
	\item[Generar conjuntos frecuentes:] \code{import\_frequent\_builds.py}
\end{description}

\section{Instalación y ejecución del proyecto}
Dado que el proyecto se ha desarrollado usando el sistema operativo Ubuntu, las instrucciones de instalación están orientadas a ese sistema.

\subsection{Requisitos previos}
Antes de comenzar con la instalación del proyecto, el sistema tiene que tener instaladas las siguientes utilidades:
\begin{itemize}
	\item \hrefFootnote{https://docs.anaconda.com/anaconda/install/index.html}{Anaconda}
	\item \hrefFootnote{https://docs.mongodb.com/manual/administration/install-community/}{MongoDB}
	\item \hrefFootnote{https://git-scm.com/downloads}{Git}
\end{itemize}

También es necesario tener una clave de acceso a la API de Riot Games para poder hacer uso de la misma. Para ello hay que crearse una cuenta de desarrollo en \url{https://developer.riotgames.com/} y seguir las instrucciones de \url{https://developer.riotgames.com/docs/portal}.

\subsection{Instalación}\label{sec:instalacion}
La forma más cómoda de obtener el código del proyecto es mediante \textit{git}, para ello usar el siguiente comando:
\begin{lstlisting}
	$ git clone <url_del_repositorio>
\end{lstlisting}
Siendo \url{https://github.com/IvanBeke/TFM} la URL del repositorio.

Ya con el proyecto descargado, el siguiente paso es instalar las dependencias. Para ello hay que ejecutar el comando:
\begin{lstlisting}
	$ pip install -r scripts and notebooks/requirementes.txt
\end{lstlisting}
Para la aplicación se recomienda usar un entorno virtual para tener mejor control sobre las dependencias exactas. Para crearlo y activarlo:
\begin{lstlisting}
	$ conda create --name django
	$ conda activate django
\end{lstlisting}
Una vez está el entorno en uso se instalan las dependencias con el comando:
\begin{lstlisting}
	$ pip install -r web/requirementes.txt
\end{lstlisting}

\subsubsection{Variables de entorno}
Para el uso de \textit{notebooks} se necesita crear un fichero \code{api\_key.txt}, en el directorio \code{scripts and notebooks} que contiene la clave de acceso a la API.

Para la aplicación web, hay que crear un fichero \code{.env}, en el directorio \textit{web} que contenga las variables de entorno. También se pueden definir en el propio sistema. El listado de variables a las que hay que dar un valor es el siguiente:
\begin{itemize}
	\item \code{RIOT\_API\_KEY}: clave de acceso a la API.
	\item \code{RIOT\_API\_REGION}: región sobre la que se van a ejecutar las peticiones.
	\item \code{SECRET\_KEY}: clave que usa \textit{Django} para seguridad interna.
	\item \code{DEBUG}: define si está activado el modo depuración de la aplicación.
\end{itemize}

\subsection{Ejecución}
Partiendo del directorio raíz se muestra como ejecutar los \textit{notebooks} o la aplicación web.
\subsubsection{Notebooks}
\begin{lstlisting}
	$ cd scripts and notebooks
	$ jupyter notebook
\end{lstlisting}
\subsubsection{Aplicación web}
\begin{lstlisting}
	$ cd web
	$ python manage.py runserver
\end{lstlisting}
Con eso se lanzaría la aplicación web. Para la ejecución de las distintas fases del proceso de recopilación y entrenamiento se ejecutarían los siguientes comandos:
\begin{lstlisting}
	$ cd web
	$ python manage.py import_summoners
	$ python manage.py import_matches
	$ python manage.py add_position
	$ python manage.py generate_transaction
	$ python manage.py generate_frequent_builds
\end{lstlisting}

%\section{Pruebas del sistema}
