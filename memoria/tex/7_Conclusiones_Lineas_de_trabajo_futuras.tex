\capitulo{7}{Conclusiones y Líneas de trabajo futuras}

\section{Conclusiones}


\section{Líneas de trabajo futuras}
En el estado actual del proyecto la aplicación puede proporcionar una gran ayuda a nuevos jugadores, a la vez que puede permitir a los profesionales analizar el estado actual del juego. Pero hay varios puntos que pueden mejorarse para ofrecer una herramienta más completa, que son los siguientes:
\begin{itemize}
	\item Tener en cuenta el orden de compra de cada objeto además del estado final.
	\item Se podrían usar algoritmos de \textit{clustering} para obtener grupos de objetos similares y proporcionar alternativas a los usuario sobre los conjuntos existentes.
	\item Hay objetos que a pesar de ser útiles, podrían no aparecer por ser muy dependientes de la situación, por ejemplo, un campeón enemigo concreto, se podría incorporar una sección con todos los objetos y mostrar situaciones concretas donde conviene comprarlos.
	\item A la hora de generar las transacciones no se tiene en cuenta el parche en el que han sido almacenadas las partidas, en futuras versiones habría que controlarlo para asegurar la información más correcta.
	\item Actualmente solo se recopilan datos del servidor de Europa, una buena forma de ampliar la cantidad de datos sería expandir la reopilación al resto de servidores (américa y asia).
\end{itemize}