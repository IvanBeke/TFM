\capitulo{7}{Conclusiones y Líneas de trabajo futuras}

\section{Conclusiones}
Habiendo terminado el desarrollo del proyecto, me siento satisfecho con el producto conseguido y considero que se han cumplido los objetivos propuestos al inicio. Como resultado se ha obtenido un proceso capaz de recopilar y analizar partidas, capaz de obtener información útil que mostrar de una forma intuitiva en una aplicación web.

He conseguido aplicar varias técnicas aprendidas durante el desarrollo de máster y me he dado cuenta de muchas consideraciones a tener en mente siempre que me encuentre en el desarrollo de proyectos que estén relacionados con obtención masiva de datos.

Para terminar, me siento contento con el trabajo que he realizado estos meses, ha sido una experiencia positiva y enriquecedora de la que estoy seguro voy a ser capaz de aplicar lo aprendido en mi futuro.

\section{Líneas de trabajo futuras}
En el estado actual del proyecto, la aplicación puede proporcionar una gran ayuda a nuevos jugadores, a la vez que puede permitir a los profesionales analizar el estado actual del juego. Pero hay varios puntos que pueden mejorarse para ofrecer una herramienta más completa, algunos de los cuales serían:
\begin{itemize}
	\item Tener en cuenta el orden de compra de cada objeto además del estado final.
	\item Se podrían usar algoritmos de \textit{clustering} para obtener grupos de objetos similares y proporcionar alternativas a los usuario sobre los conjuntos existentes.
	\item Hay objetos que a pesar de ser útiles, podrían no aparecer por ser muy dependientes de la situación, por ejemplo, un campeón enemigo concreto, se podría incorporar una sección con todos los objetos y mostrar situaciones concretas donde conviene comprarlos.
	\item A la hora de generar las transacciones no se tiene en cuenta la versión del juego en la que han sido jugadas las partidas, lo que puede derivar en que las transacciones mezclen partidas que se hayan jugado en distintas actualizaciones. En futuras versiones de la aplicación habría que controlarlo para asegurar la información más correcta.
	\item Actualmente solo se recopilan datos del servidor de Europa, una buena forma de ampliar la cantidad de datos sería expandir la recopilación al resto de servidores (América y Asia).
\end{itemize}