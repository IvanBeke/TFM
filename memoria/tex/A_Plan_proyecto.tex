\apendice{Plan de Proyecto Software}

\section{Introducción}

\section{Planificación temporal}

\subsection{Sprint 0}

Sprint dedicado a tareas logísticas y preparación del inicio del proyecto.

\begin{itemize}
    \item Creación del repositorio de código.
    \item Solicitud de una clave permanente para la API a la desarrolladora del juego.
    \item Pruebas con la API para comprobar la existencia de los datos necesarios.
    \item Reunión para formalizar el inicio del desarrollo del proyecto.
\end{itemize}

\subsection{Sprint 1}

Primer sprint de desarrollo. Dedicado a la extracción de jugadores.

\begin{itemize}
    \item Documentación Sprint 0.
    \item Echar un ojo a wrappers existentes de la API para comprobar su viabilidad de uso.
    \item Desarrollo de un notebook para la extracción de jugadores, teniendo en cuenta límite de peticiones y tolerancia a fallos.
\end{itemize}

\subsection{Sprint 2}

\begin{itemize}
    \item Documentación Sprint 1.
    \item Modificar extracción de jugadores para añadir identificador de cuenta, necesario para el siguiente paso de recuperación de partidas.
    \item Desarrollo de un notebook para extraer las partidas de los jugadores recuperador previamente.
    \item Crear biblioteca con funciones comunes para realizar peticiones y guardar estados de ejecución.
\end{itemize}

\subsection{Sprint 3}

\begin{itemize}
    \item Documentación Sprint 2.
    \item Refactorizar un método para formar las url completas a las que llamar a la API.
    \item Partiendo de los datos de partidas del sprint anterior, localizar y extraer los objetos que ha comprado para personaje.
    \item Crear el formato final con los inputs para el algoritmo apriori.
\end{itemize}

\section{Estudio de viabilidad}

\subsection{Viabilidad económica}

\subsection{Viabilidad legal}


