\apendice{Plan de Proyecto Software}

\section{Introducción}

En este apartado se va a exponer como se ha llevado a cabo la planificación del proyecto, además de los avances logrados entre cada iteración y reunión.

\section{Planificación temporal}

Durante el desarrollo del proyecto se ha seguido una versión simplificada de \textit{Scrum}. Se han usando \textit{spints} para revisar los avances, con duración de una semana. Al final de cada \textit{sprint} se realiza una reunión por videoconferencia para ver los avances logrados durante el mismo y planificar las tareas del siguiente.

El inicio oficial del proyecto fue el día 19 de mayo, con la reunión del Sprint 0 \ref{sprint-0}, habiendo realizado las tareas de ese sprint previamente.

\subsection{Sprint 0}\label{sprint-0}

Sprint dedicado a tareas logísticas y preparación del inicio del proyecto.

\begin{itemize}
    \item Creación del repositorio de código.
    \item Solicitud de una clave permanente para la API a la desarrolladora del juego.
    \item Pruebas con la API para comprobar la existencia de los datos necesarios.
    \item Reunión para formalizar el inicio del desarrollo del proyecto.
\end{itemize}

\subsection{Sprint 1}

Primer sprint de desarrollo. Dedicado a la extracción de jugadores. Desde 20/05/2021 hasta el 26/05/2021.

\begin{itemize}
    \item Documentación Sprint 0.
    \item Echar un ojo a \textit{wrappers} existentes de la API para comprobar su viabilidad de uso.
    \item Desarrollo de un \textit{notebook} para la extracción de jugadores, teniendo en cuenta límite de peticiones y tolerancia a fallos.
\end{itemize}

\subsection{Sprint 2}

Desde 27/05/2021 hasta el 02/06/2021.

\begin{itemize}
    \item Documentación Sprint 1.
    \item Modificar extracción de jugadores para añadir identificador de cuenta, necesario para el siguiente paso de recuperación de partidas.
    \item Desarrollo de un \textit{notebook} para extraer las partidas de los jugadores recuperador previamente.
    \item Crear biblioteca con funciones comunes para realizar peticiones y guardar estados de ejecución.
\end{itemize}

\subsection{Sprint 3}

Desde 03/06/2021 hasta el 09/06/2021.

\begin{itemize}
    \item Documentación Sprint 2.
    \item Partiendo de los datos de partidas del sprint anterior, localizar y extraer los objetos que se han comprado para cada campeón dentro de la partida.
    \item Crear el formato final con las entradas para el algoritmo apriori.
\end{itemize}

\subsection{Sprint 4}

Desde 10/06/2021 hasta el 16/06/2021. Por conflicto con otras asignaturas durante esta semana se realizaron menos tareas.

\begin{itemize}
    \item Documentación Sprint 3.
    \item Escribir la introducción de la memoria.
    \item Presentación sobre funcionamiento y conceptos del juego a los tutores.
\end{itemize}

\subsection{Sprint 5}

Desde 17/06/2021 hasta el 30/06/2021. La duración de este sprint se alargó a dos semanas por un problema de salud que impidió realizar avances significativos durante la primera.

\begin{itemize}
    \item Correcciones en la memoria.
    \item Escribir objetivos del proyecto.
    \item Investigar alternativas del algoritmo apriori.
    \item Buscar y escoger una biblioteca con las implementaciones del algoritmo a utilizar y sus alternativas.
    \item Aprendizaje de Django.
    \item Inicio del desarrollo de la aplicación web.
\end{itemize}

\subsection{Sprint 6}

Desde 01/06/2021 hasta el 07/07/2021. Por conflicto con otras asignaturas durante esta semana se realizaron menos tareas.

\begin{itemize}
    \item Desarrollo de un notebook en el que probar los algoritmos de conjuntos frecuentes.
    \item Selección del algoritmo a usar para obtener los resultados que mostrar en la aplicación.
\end{itemize}

\subsection{Sprint 7}

Desde 08/07/2021 hasta 14/07/2021.

\begin{itemize}
    \item Instalación de base de datos MongoDB.
    \item Guardar ejecución del algoritmo en MongoDB.
    \item Listado de campeones en la aplicación web.
    \item Al seleccionar un campeón mostrar sus conjuntos de objetos frecuentes.
\end{itemize}

\subsection{Sprint 8}

\begin{itemize}
    \item Documentación sprints 4, 5, 6 y 7.
\end{itemize}

\section{Estudio de viabilidad}

\subsection{Viabilidad económica}

\subsection{Viabilidad legal}


