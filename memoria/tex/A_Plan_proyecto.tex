\apendice{Plan de Proyecto Software}

\section{Introducción}

En este apartado se va a exponer como se ha llevado a cabo la planificación del proyecto, además de los avances logrados entre cada iteración y reunión.

\section{Planificación temporal}

Durante el desarrollo del proyecto se ha seguido una versión simplificada de \textit{Scrum}. Se han usado \textit{sprints} para revisar los avances, con duración de una semana. Al final de cada \textit{sprint} se realiza una reunión por videoconferencia para ver los avances logrados durante el mismo y planificar las tareas del siguiente.

El inicio oficial del proyecto fue el día 19 de mayo, con la reunión del Sprint 0 \ref{sprint-0}, habiendo realizado las tareas de ese sprint previamente.

\subsection{Sprint 0}\label{sprint-0}

Sprint dedicado a tareas logísticas y preparación del inicio del proyecto.

\begin{itemize}
    \item Crear el repositorio de código.
    \item Solicitar una clave permanente para la API a la desarrolladora del juego.
    \item Comprobaren la API la existencia de los datos necesarios.
    \item Formalizar el inicio del desarrollo del proyecto mediante una reunión con los tutores.
\end{itemize}

\subsection{Sprint 1}

Primer sprint de desarrollo. Dedicado a la extracción de jugadores. Desde 20/05/2021 hasta el 26/05/2021.

\begin{itemize}
    \item Documentar el Sprint 0.
    \item Echar un ojo a \textit{wrappers} existentes de la API para comprobar su viabilidad de uso.
    \item Desarrollar un \textit{notebook} para la extracción de jugadores, teniendo en cuenta límite de peticiones y tolerancia a fallos.
\end{itemize}

\subsection{Sprint 2}

Desde 27/05/2021 hasta el 02/06/2021.

\begin{itemize}
    \item Documentar el Sprint 1.
    \item Modificar extracción de jugadores para añadir identificador de cuenta, necesario para el siguiente paso de recuperación de partidas.
    \item Desarrollar un \textit{notebook} para extraer las partidas de los jugadores recuperados previamente.
    \item Crear biblioteca con funciones comunes para realizar peticiones y guardar estados de ejecución.
\end{itemize}

\subsection{Sprint 3}

Desde 03/06/2021 hasta el 09/06/2021.

\begin{itemize}
    \item Documentar el Sprint 2.
    \item Partiendo de los datos de partidas del sprint anterior, localizar y extraer los objetos que se han comprado para cada campeón dentro de la partida.
    \item Crear el formato final con las entradas para el algoritmo apriori.
\end{itemize}

\subsection{Sprint 4}

Desde 10/06/2021 hasta el 16/06/2021. Por conflicto con otras asignaturas durante esta semana se realizaron menos tareas.

\begin{itemize}
    \item Documentar Sprint 3.
    \item Escribir la introducción de la memoria.
    \item Realizar una presentación sobre el funcionamiento y conceptos del juego a los tutores.
\end{itemize}

\subsection{Sprint 5}

Desde 17/06/2021 hasta el 30/06/2021. La duración de este sprint se alargó a dos semanas por un problema de salud que impidió realizar avances significativos durante la primera.

\begin{itemize}
    \item Realizar correcciones en la memoria.
    \item Escribir objetivos del proyecto.
    \item Investigar alternativas del algoritmo apriori.
    \item Buscar y escoger una biblioteca con las implementaciones del algoritmo a utilizar y sus alternativas.
    \item Empezar con el aprendizaje de Django.
    \item Iniciar el desarrollo de la aplicación web.
\end{itemize}

\subsection{Sprint 6}

Desde 01/06/2021 hasta el 07/07/2021. Por conflicto con otras asignaturas durante esta semana se realizaron menos tareas.

\begin{itemize}
    \item Desarrollar un notebook en el que probar los algoritmos de conjuntos frecuentes.
    \item Seleccionar el algoritmo a usar para obtener los resultados que mostrar en la aplicación.
\end{itemize}

\subsection{Sprint 7}

Desde 08/07/2021 hasta 14/07/2021.

\begin{itemize}
    \item Instalar una base de datos MongoDB.
    \item Guardar ejecución del algoritmo en MongoDB.
    \item Crear una vista con el listado de campeones en la aplicación web.
    \item Crear una vista con en la que mostrar los conjuntos de objetos frecuentes de un campeón seleccionado.
\end{itemize}

\subsection{Sprint 8}

Desde 15/07/2021 hasta 21/07/2021.

\begin{itemize}
    \item Documentar los sprints 4, 5, 6 y 7.
    \item Incorporar el \textit{notebook} de extracción de jugadores a la aplicación web.
 	\item Incorporar el \textit{notebook} de extracción de partidas a la aplicación web.
 	\item Añadir buscador de campeones.
 	\item Añadir la ficha del campeón junto a sus objetos frecuentes.
 	\item Mostrar un mensaje cuando no existen datos para un campeón.
\end{itemize}

\subsection{Sprint 9}

Desde 22/07/2021 hasta 28/07/2021.

\begin{itemize}
	\item Documentar los sprints 8 y 9.
	\item Extraer partidas de forma masiva.
    \item Incorporar el \textit{notebook} de extracción de detalles de partidas a la aplicación web.
    \item Incorporar el \textit{notebook} de extracción de ejecución de algoritmos a la aplicación web.
    \item Escribir los conceptos teóricos sobre el juego.
\end{itemize}

\subsection{Sprint 10}

Desde 29/07/2021 hasta 04/08/2021.

\begin{itemize}
	\item Modificar la generación de transacciones para agrupar también por posición.
	\item Añadir pestañas a la vista de conjuntos frecuentes, una por posición.
	\item Cambiar el porcentaje de elección en texto por una barra de progreso.
\end{itemize}

\subsection{Sprint 11}

Desde 05/08/2021 hasta 18/08/2021.

\begin{itemize}
	\item Realizar correcciones sobre la memoria.
	\item Intentar añadir panel de administrador para lanzar las tareas de la pipeline desde la interfaz.
\end{itemize}

\subsection{Sprint 12}

Desde 19/08/2021 hasta 25/08/2021.

\begin{itemize}
	\item Realizar correcciones sobre la memoria.
	\item Escribir la memoria.
\end{itemize}

\section{Estudio de viabilidad}

\subsection{Viabilidad económica}

\subsection{Viabilidad legal}


