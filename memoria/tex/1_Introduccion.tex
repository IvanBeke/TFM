\capitulo{1}{Introducción}

Las competiciones de deportes electrónicos, o \textit{esports}, al igual que los deportes tradicionales, mueven grandes cantidades de dinero a la vez que atraen a un número muy elevado de espectadores a sus retransmisiones.

Por lo general los juegos de los que se realizan competiciones son gratuitos, por lo que cualquier persona puede adentrarse en ese mundillo para pasar un rato entretenido, o ponerse la meta de llegar a ser profesional.

Sin embargo, esto es un objetivo complicado por la gran cantidad de horas necesarias para conseguir las capacidades necesarias para ser profesional. Además, el rango de edad en el que más necesario dedicar más horas de practica coincide con etapas de escolarización todavía obligatorias, pudiendo crear conflicto de intereses.

En ambos ámbitos, profesional y casual, se genera constantemente una gran cantidad de datos, tomando la forma de un registro de partidas jugadas. Para este trabajo me he propuesto analizar ese histórico de partidas en uno de los \textit{esports} más predominantes del momento, League of Legends, un videojuego dentro del tipo MOBA (\textit{Multiplayer Online Battle Arena}).

Mediante el uso de aprendizaje no supervisado, se puede extraer conocimiento del juego y ponerlo a disposición de las personas que empiezan a jugar y hacer más fácil esta entrada. Además de servir de ayuda en el ámbito profesional, para facilitar la preparación de un equipo ante una partida de competición.

\hfill
Ideas 
\begin{itemize}
    \item Importancia del mercado de los deportes electrónicos.
    \item Cifras de beneficios
    \item Cifras de espectadores
    \item https://newzoo.com/insights/trend-reports/newzoos-global-esports-live-streaming-market-report-2021-free-version/
    \item https://dotesports.com/league-of-legends/news/league-of-legends-reportedly-generated-1-75-billion-in-revenue-in-2020
    \item https://techacake.com/league-of-legends-player-count/
\end{itemize}