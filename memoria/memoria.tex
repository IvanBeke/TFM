\documentclass[a4paper,12pt,twoside]{memoir}

\setsecnumdepth{subsection}

% Castellano
\usepackage[spanish,es-tabla]{babel}
\selectlanguage{spanish}
\usepackage[utf8]{inputenc}
\usepackage[T1]{fontenc}
\usepackage{lmodern} % Scalable font
\usepackage{microtype}
\usepackage{placeins}
\usepackage{dirtree}
\usepackage{listings}
\usepackage[hyphens]{url}
\usepackage{booktabs}

\RequirePackage{booktabs}
\RequirePackage[table]{xcolor}
\RequirePackage{xtab}
\RequirePackage{multirow}

\lstset{%
	basicstyle=\small\ttfamily
}

% Links
\usepackage[colorlinks]{hyperref}
\hypersetup{
	allcolors = {red},
	breaklinks = true,
}

% Ecuaciones
\usepackage{amsmath}

% Rutas de fichero / paquete
\newcommand{\ruta}[1]{{\sffamily #1}}

% Párrafos
\nonzeroparskip

% Huérfanas y viudas
\widowpenalty100000
\clubpenalty100000

% Evitar solapes en el header
\nouppercaseheads

% Imagenes
\usepackage{graphicx}
\newcommand{\imagen}[2]{
	\begin{figure}[!h]
		\centering
		\includegraphics[width=0.9\textwidth]{#1}
		\caption{#2}\label{fig:#1}
	\end{figure}
	\FloatBarrier
}

\newcommand{\imagenflotante}[2]{
	\begin{figure}%[!h]
		\centering
		\includegraphics[width=0.9\textwidth]{#1}
		\caption{#2}\label{fig:#1}
	\end{figure}
}



% El comando \figura nos permite insertar figuras comodamente, y utilizando
% siempre el mismo formato. Los parametros son:
% 1 -> Porcentaje del ancho de página que ocupará la figura (de 0 a 1)
% 2 --> Fichero de la imagen
% 3 --> Texto a pie de imagen
% 4 --> Etiqueta (label) para referencias
% 5 --> Opciones que queramos pasarle al \includegraphics
% 6 --> Opciones de posicionamiento a pasarle a \begin{figure}
\newcommand{\figuraConPosicion}[6]{%
  \setlength{\anchoFloat}{#1\textwidth}%
  \addtolength{\anchoFloat}{-4\fboxsep}%
  \setlength{\anchoFigura}{\anchoFloat}%
  \begin{figure}[#6]
    \begin{center}%
      \Ovalbox{%
        \begin{minipage}{\anchoFloat}%
          \begin{center}%
            \includegraphics[width=\anchoFigura,#5]{#2}%
            \caption{#3}%
            \label{#4}%
          \end{center}%
        \end{minipage}
      }%
    \end{center}%
  \end{figure}%
}

%
% Comando para incluir imágenes en formato apaisado (sin marco).
\newcommand{\figuraApaisadaSinMarco}[5]{%
  \begin{figure}%
    \begin{center}%
    \includegraphics[angle=90,height=#1\textheight,#5]{#2}%
    \caption{#3}%
    \label{#4}%
    \end{center}%
  \end{figure}%
}
% Para las tablas
\newcommand{\otoprule}{\midrule [\heavyrulewidth]}
%
% Nuevo comando para tablas pequeñas (menos de una página).
\newcommand{\tablaSmall}[5]{%
 \begin{table}[h]
  \begin{center}
   \rowcolors {2}{gray!35}{}
   \begin{tabular}{#2}
    \toprule
    #4
    \otoprule
    #5
    \bottomrule
   \end{tabular}
   \caption{#1}
   \label{tabla:#3}
  \end{center}
 \end{table}
}

%
% Nuevo comando para tablas pequeñas (menos de una página).
\newcommand{\tablaSmallSinColores}[5]{%
 \begin{table}[H]
  \begin{center}
   \begin{tabular}{#2}
    \toprule
    #4
    \otoprule
    #5
    \bottomrule
   \end{tabular}
   \caption{#1}
   \label{tabla:#3}
  \end{center}
 \end{table}
}

\newcommand{\tablaApaisadaSmall}[5]{%
\begin{landscape}
  \begin{table}
   \begin{center}
    \rowcolors {2}{gray!35}{}
    \begin{tabular}{#2}
     \toprule
     #4
     \otoprule
     #5
     \bottomrule
    \end{tabular}
    \caption{#1}
    \label{tabla:#3}
   \end{center}
  \end{table}
\end{landscape}
}

%
% Nuevo comando para tablas grandes con cabecera y filas alternas coloreadas en gris.
\newcommand{\tabla}[6]{%
  \begin{center}
    \tablefirsthead{
      \toprule
      #5
      \otoprule
    }
    \tablehead{
      \multicolumn{#3}{l}{\small\sl continúa desde la página anterior}\\
      \toprule
      #5
      \otoprule
    }
    \tabletail{
      \hline
      \multicolumn{#3}{r}{\small\sl continúa en la página siguiente}\\
    }
    \tablelasttail{
      \hline
    }
    \bottomcaption{#1}
    \rowcolors {2}{gray!35}{}
    \begin{xtabular}{#2}
      #6
      \bottomrule
    \end{xtabular}
    \label{tabla:#4}
  \end{center}
}

%
% Nuevo comando para tablas grandes con cabecera.
\newcommand{\tablaSinColores}[6]{%
  \begin{center}
    \tablefirsthead{
      \toprule
      #5
      \otoprule
    }
    \tablehead{
      \multicolumn{#3}{l}{\small\sl continúa desde la página anterior}\\
      \toprule
      #5
      \otoprule
    }
    \tabletail{
      \hline
      \multicolumn{#3}{r}{\small\sl continúa en la página siguiente}\\
    }
    \tablelasttail{
      \hline
    }
    \bottomcaption{#1}
    \begin{xtabular}{#2}
      #6
      \bottomrule
    \end{xtabular}
    \label{tabla:#4}
  \end{center}
}

%
% Nuevo comando para tablas grandes sin cabecera.
\newcommand{\tablaSinCabecera}[5]{%
  \begin{center}
    \tablefirsthead{
      \toprule
    }
    \tablehead{
      \multicolumn{#3}{l}{\small\sl continúa desde la página anterior}\\
      \hline
    }
    \tabletail{
      \hline
      \multicolumn{#3}{r}{\small\sl continúa en la página siguiente}\\
    }
    \tablelasttail{
      \hline
    }
    \bottomcaption{#1}
  \begin{xtabular}{#2}
    #5
   \bottomrule
  \end{xtabular}
  \label{tabla:#4}
  \end{center}
}

\newcommand{\hrefFootnote}[2]{%
	\href{#1}{#2}\footnote{\url{#1}}
}

\newcommand{\code}[1]{\texttt{#1}}


\definecolor{cgoLight}{HTML}{EEEEEE}
\definecolor{cgoExtralight}{HTML}{FFFFFF}

%
% Nuevo comando para tablas grandes sin cabecera.
\newcommand{\tablaSinCabeceraConBandas}[5]{%
  \begin{center}
    \tablefirsthead{
      \toprule
    }
    \tablehead{
      \multicolumn{#3}{l}{\small\sl continúa desde la página anterior}\\
      \hline
    }
    \tabletail{
      \hline
      \multicolumn{#3}{r}{\small\sl continúa en la página siguiente}\\
    }
    \tablelasttail{
      \hline
    }
    \bottomcaption{#1}
    \rowcolors[]{1}{cgoExtralight}{cgoLight}

  \begin{xtabular}{#2}
    #5
   \bottomrule
  \end{xtabular}
  \label{tabla:#4}
  \end{center}
}


\graphicspath{ {./img/} }

% Capítulos
\chapterstyle{bianchi}
\newcommand{\capitulo}[2]{
	\setcounter{chapter}{#1}
	\setcounter{section}{0}
	\chapter*{#2}
	\addcontentsline{toc}{chapter}{#1. #2}
	\markboth{#2}{#2}
}

% Apéndices
\renewcommand{\appendixname}{Apéndice}
\renewcommand*\cftappendixname{\appendixname}

\newcommand{\apendice}[1]{
	%\renewcommand{\thechapter}{A}
	\chapter{#1}
}

\renewcommand*\cftappendixname{\appendixname\ }

% Formato de portada
\makeatletter
\usepackage{xcolor}
\newcommand{\tutor}[1]{\def\@tutor{#1}}
\newcommand{\cotutor}[1]{\def\@cotutor{#1}}
\newcommand{\course}[1]{\def\@course{#1}}
\definecolor{cpardoBox}{HTML}{E6E6FF}
\def\maketitle{
	\null
	\thispagestyle{empty}
	% Cabecera ----------------
	\begin{center}%
		{\noindent\LARGE Universidades de Burgos, León y Valladolid}\vspace{.5cm}%

		{\noindent\large Máster universitario}\vspace{.5cm}%

		{\noindent\LARGE \textbf{Inteligencia de Negocio y Big~Data en Entornos Seguros}}\vspace{.5cm}%
	\end{center}%

	\begin{center}%
		\includegraphics[height=3cm]{img/escudoUBU} \hspace{1cm}
		\includegraphics[height=3cm]{img/escudoUVA} \hspace{1cm}
		\includegraphics[height=3cm]{img/escudoULE} \vspace{1cm}%
	\end{center}%

	\vfill
	% Título proyecto y escudo informática ----------------
	\colorbox{cpardoBox}{%
		\begin{minipage}{.9\textwidth}
			\vspace{.4cm}\large
			\begin{center}
				\textbf{TFM del Máster Inteligencia de Negocio y Big Data en Entornos Seguros}\vspace{.6cm}\\
				\textbf{\Large\@title{}}
			\end{center}
			\vspace{.2cm}
		\end{minipage}

	}%
	\vfill
	% Datos de alumno, curso y tutores ------------------
	\begin{center}%
		{%
			\noindent\Large
			Presentado por \@author{}\\
			en la Universidad de Burgos --- \@date{}\\[1em]
			Tutores: Dr. \@tutor{} y\\\hspace{3.7em}Dr. \@cotutor{}
		}%
	\end{center}%
	\null
	\cleardoublepage
}
\makeatother

\newcommand{\nombre}{Iván Iglesias Cuesta} %%% cambio de comando
\newcommand{\titulo}{Uso de técnicas de aprendizaje no supervisado para la ayuda en videojuegos tipo MOBA}
\newcommand{\nombretutor}{José Francisco Díez Pastor}
\newcommand{\nombrecotutor}{César Ignacio García Osorio}

% Datos de portada
\title{\titulo}
\author{\nombre}
\tutor{\nombretutor}
\cotutor{\nombrecotutor}
\date{\today}

\begin{document}

\maketitle


\newpage\null\thispagestyle{empty}\newpage


%%%%%%%%%%%%%%%%%%%%%%%%%%%%%%%%%%%%%%%%%%%%%%%%%%%%%%%%%%%%%%%%%%%%%%%%%%%%%%%%%%%%%%%%
\thispagestyle{empty}


\noindent
\begin{center}%
	{\noindent\Huge Universidades de Burgos, León y Valladolid}\vspace{.5cm}%

\begin{center}%
	\includegraphics[height=3cm]{img/escudoUBU} \hspace{1cm}
	\includegraphics[height=3cm]{img/escudoUVA} \hspace{1cm}
	\includegraphics[height=3cm]{img/escudoULE} \vspace{1cm}%
\end{center}%

	{\noindent\Large \textbf{Máster universitario en Inteligencia de Negocio y Big~Data en Entornos Seguros}}\vspace{.5cm}%
\end{center}%



\noindent D. \nombretutor, profesor del departamento de Ingeniería Informática, área de Lenguajes y Sistemas Informáticos.

\noindent D. \nombrecotutor, profesor del departamento de Ingeniería Informática, área de Lenguajes y Sistemas Informáticos.

\noindent Exponen:

\noindent Que el alumno D. \nombre, con DNI 45573756S, ha realizado el Trabajo final de Máster en Inteligencia de Negocio y Big Data en Entornos Seguros
          titulado <<\titulo>>.

\noindent Y que dicho trabajo ha sido realizado por el alumno bajo la dirección del que suscribe, en virtud de lo cual se autoriza su presentación y defensa.

\begin{center} %\large
En Burgos, {\large \today}
\end{center}

\vfill\vfill\vfill

% Author and supervisor
\begin{minipage}{0.45\textwidth}
\begin{flushleft} %\large
Vº. Bº. del Tutor:\\[2cm]
D. \nombretutor
\end{flushleft}
\end{minipage}
\hfill
\begin{minipage}{0.45\textwidth}
\begin{flushleft} %\large
Vº. Bº. del Tutor:\\[2cm]
D. \nombrecotutor
\end{flushleft}
\end{minipage}
\hfill

\vfill

% para casos con solo un tutor comentar lo anterior
% y descomentar lo siguiente
%Vº. Bº. del Tutor:\\[2cm]
%D. nombre tutor


\newpage\null\thispagestyle{empty}\newpage




\frontmatter

% Abstract en castellano
\renewcommand*\abstractname{Resumen}
\begin{abstract}
Desde hace unos años, el mundo de los deportes electrónicos, o \textit{eSports}, no para de crecer. Con la gran cantidad de jugadores dedicados a estos títulos, como entretenimiento y como carrera profesional, se generan constantemente datos que pueden explotarse para ayudar a nuevos jugadores a iniciarse, y analizar las propias tendencias internas del juego. Generalmente la barrera de entrada en estos juegos es elevada, al ser necesario tener conocimientos de sus mecánicas, para poder desenvolverse en partida y para disfrutar de la experiencia, por lo que los primeros pasos suelen ser frustrantes. Por ello, en este trabajo, usando de ejemplo \textit{League of Legends} (uno de los \textit{eSports} más populares), se plantea un proceso de recopilación, tratamiento y obtención de conocimiento, a partir de las partidas jugadas. Adicionalmente se ha creado una aplicación web para mostrar los resultados obtenidos del análisis de partidas.
\end{abstract}

\renewcommand*\abstractname{Descriptores}
\begin{abstract}
MOBA, League of Legends, Riot Games, aprendizaje no supervisado, conjuntos frecuentes, ETL, Django, MongoDB.
\end{abstract}

\clearpage

% Abstract en inglés
\renewcommand*\abstractname{Abstract}
\begin{abstract}
For a number of years now, the world of eSports has been growing steadily. With the large number of players dedicated to these titles, both as entertainment and as a career, data is constantly being generated that can be exploited to help new players get started, and to analyse the game's own internal trends. Generally, the entry barrier in these games is high, as it is necessary to have knowledge of the game mechanics to be able to play the game and enjoy the experience, so the first steps are often frustrating. For this reason, in this project, using the example of League of Legends (one of the most popular eSports title), a pipeline for collecting, processing and obtaining knowledge from past games is proposed. Additionally, a web application is created to display the results obtained from the games analysis.
\end{abstract}

\renewcommand*\abstractname{Keywords}
\begin{abstract}
MOBA, League of Legends, Riot Games, unsupervised learning, frequent itemsets, ETL, Django, MongoDB.
\end{abstract}

\clearpage

% Indices
\tableofcontents

\clearpage

\listoffigures

\clearpage

\listoftables
\clearpage

\mainmatter

\addcontentsline{toc}{part}{Memoria}
\part*{Memoria}

\capitulo{1}{Introducción}

Las competiciones de deportes electrónicos, o \textit{eSports}, al igual que los deportes tradicionales, mueven grandes cantidades de dinero a la vez que atraen a un número muy elevado de espectadores a sus retransmisiones.

Por lo general los juegos de los que se realizan competiciones son gratuitos, por lo que cualquier persona puede adentrarse en ese mundillo para pasar un rato entretenido, o ponerse la meta de llegar a ser profesional.

Sin embargo, llegar a ser profesional es un objetivo complicado por la gran cantidad de horas necesarias para conseguir las capacidades necesarias para llegar a esos niveles de competencia. Además, el rango de edad en el que es más necesario dedicar más horas coincide con etapas de escolarización todavía obligatorias, pudiendo crear conflicto de intereses.

En ambos ámbitos, profesional y casual, se genera constantemente una gran cantidad de datos, tomando la forma de un registro de partidas jugadas. Para este trabajo me he propuesto analizar ese histórico de partidas en uno de los \textit{eSports} más predominantes del momento, \textit{League of Legends}, un videojuego dentro del tipo MOBA (\textit{Multiplayer Online Battle Arena}).

Por dar un poco de contexto a la importancia de este título en el mundo actual, en el año 2020 el juego obtuvo un beneficio de 1.750 millones de dólares~\cite{michael_2021}. En comparación con otros mercados más tradicionales, en el mismo año, Inditex obtuvo 1.100 millones de euros de beneficios~\cite{cortizo_2021}.

Además de tener números impresionantes en cuanto a beneficios, se estima que, en lo que llevamos de 2021, cada mes juegan de media, hasta 115 millones de personas~\cite{lol_players}. Este elevado número de jugadores genera una gran cantidad de datos, en forma de partidas jugadas, que se pueden usar para analizar el juego y extraer conocimiento sobre cómo se juega.

Mediante el uso de aprendizaje no supervisado, se puede extraer conocimiento del juego y ponerlo a disposición de las personas que empiezan a jugar y hacer más fácil esta entrada. Además de servir de ayuda en el ámbito profesional, para facilitar la preparación de un equipo ante una partida de competición.

%\hfill \break
%Ideas
%\begin{itemize}
%    \item Importancia del mercado de los deportes electrónicos.
%    \item Cifras de beneficios
%    \item Cifras de espectadores
%    \item \url{https://www.esportmaniacos.com/business/excel-esports-inversion-20-millones/}
%    \item \url{https://www.esportmaniacos.com/comunidad/jugadores-futbol-esports-clubes/}
%    \item \url{https://newzoo.com/insights/trend-reports/newzoos-global-esports-live-streaming-market-report-2021-free-version/}
%    \item \url{https://dotesports.com/league-of-legends/news/league-of-legends-reportedly-generated-1-75-billion-in-revenue-in-2020}
%    \item \url{https://techacake.com/league-of-legends-player-count/}
%    \item \url{https://www.bdsesport.com/en/actualites/hot-news/team-bds-acquires-fc-schalke-04s-lec-slot}
%    \item \url{https://www.openbank.es/superliga-esportsg_8OTUKlqc3YybD6qYP-bygTCX2aS19UoGb5t0SSf-LflC8g_8OTUKlqc3YybD6qYP-bygTCX2aS19UoGb5t0SSf-LflC8}
%\end{itemize}
\capitulo{2}{Objetivos del proyecto}

\section{Objetivos principales}
\begin{itemize}
    \item Desarrollar un proceso ETL que sea capaz de recopilar los datos necesarios usando la API oficial de Riot Games\footnote{Riot Games es la desarrolladora de League of Legends}.
    \item Aplicar técnicas de aprendizaje no supervisado sobre los datos recopilados, en este caso algoritmos para la obtención de conjuntos frecuentes de objetos.
    \item Ser capaz de obtener conocimiento útil a partir de los datos obtenidos.
    \item Desarrollar una aplicación en la que se pueda consultar el conocimiento extraído.
    \item Que el producto final sea capaz de ayudar a los nuevos jugadores.
\end{itemize}


\section{Objetivos personales}
\begin{itemize}
    \item Aplicar lo aprendido durante el máster en un campo novedoso.
    \item Dar a conocer el mundo de los deportes electrónicos en un ambiente donde sean menos conocidos.
    \item Desarrollar un proyecto de ideación propia.
\end{itemize}
\capitulo{3}{Conceptos teóricos}

\section{Conceptos sobre \textit{League of Legends}}

\subsection{El juego}
\textit{League of Legends} es un videojuego de estrategia multijugador del tipo \textit{MOBA (Multiplayer Online Battle Arena)} lanzado en 2011 en el que dos equipos de cinco jugadores se enfrentan para destruir la base del equipo enemigo \cite{misc:como-jugar}. Cada jugador controla dentro del juego a un personaje llamado campeón, que pueden seleccionar antes de empezar a jugar, y es único entre los diez jugadores.

Los jugadores se enfrentan en un mapa \ref{fig:mapa-lol} en forma de cuadrado, con las bases de cada equipo localizadas zonas opuestas del mapa, una en la esquina inferior izquierda, y la otra en la superior derecha. Conectando cada base se encuentran tres líneas o calles, superior o \textit{top}, central o \textit{mid} e inferior o \textit{bot}. El espacio entre las calles se denomina jungla. Conectando las dos esquinas que no pertenecen a las bases se encuentra el río, que se encarga de separar el terreno del mapa dominado por cada equipo. De forma general los jugadores se reparten de la siguiente manera, uno en \textit{top}, uno en \textit{mid}, dos en \textit{bot} y el restante en la jungla.

\begin{figure}
	\centering
	\includegraphics[width=1\linewidth]{img/mapa-lol}
	\caption{Mapa de \textit{League of Legends}.}
	\label{fig:mapa-lol}
\end{figure}

Para ganar la partida hay que destruir el nexo del equipo enemigo, es la estructura que está más alejada de la base propia. Las torres son otro tipo de estructuras situadas por el mapa, que disparan a los campeones del equipo contrario. Existen tres torres por cada línea y equipo, además de dos adicionales que protegen cada nexo. Las torres se tienen que derribar en el orden que se van encontrando en cada línea, y para destruir el nexo, como mínimo, hay que haber derribado todas las torres de una calle.

La forma en la que un equipo consigue ventaja sobre el rival es mediante el oro, este se consigue de varias formas. La principal forma es asesinando a los campeones enemigos. Otra formas de conseguir oro es destruyendo las estructuras enemigas. La última forma de conseguir oro es matando monstruos y súbditos, los primeros se encuentran en la jungla, los segundos recorren las calles en oleadas. Tanto los campeones como los monstruos y súbditos vuelven a aparecen pasado un tiempo concreto, las estructuras una vez destruidas se mantienen así.

El oro permite comprar objetos, que mejoran las habilidades del campeón, haciendo que sea más sencillo derrotar a los campeones enemigos, lo que proporciona más oro para más objetos, causando un efecto bola de nieve y la eventual victoria del equipo.

Para ver estos conceptos de una forma más visual, Riot Games preparó un vídeo informativo de cara a los mundiales de 2020 para que la gente que no tuviera mucho conocimiento del juego y su funcionamiento pudiera ver y disfrutar la competición. El vídeo se titula \href{https://www.youtube.com/watch?v=ERkt_1TYlkU}{¿Así que queréis ver el Mundial? | Mundial 2020 - League of Legends}\footnote{\url{https://youtu.be/ERkt_1TYlkU}} y se encuentra disponible en YouTube.


\subsection{Campeones}
Los campeones son los diferentes personajes disponibles que un jugador puede seleccionar antes de la partida. Actualmente hay 156 disponibles, añadiéndose a la lista uno nuevo en franjas de tiempo que van desde uno a seis meses. Cada uno tiene un rol asignado que determina su forma de jugar, su función e incluso posición dentro del mapa.
\begin{itemize}
	\tightlist
	\item Tirador
	\item Apoyo
	\item Asesino
	\item Luchador
	\item Tanque
	\item Mago
\end{itemize}

Para que un campeón acabe en una categoría u otra hay que prestar atención a varios factores, entre los que se encuentran su tipo de ataque básico, el efecto de sus habilidades y la forma en la que sus estadísticas modifican las habilidades. En las secciones \ref{habilidades} y \ref{estadisticas} se explican en detalles estos conceptos.

\subsection{Habilidades}
\label{habilidades}
Cada campeón tiene un conjunto único de habilidades, una pasiva y cuatro activas. Se pueden definir como las operaciones que un jugador puede realizar para interactuar con el mapa, con otros jugadores, monstruos y súbditos.

Cada habilidad puede tener un efecto o combinar varios, los cuales se aplican sobre uno mismo, un aliado o enemigo. Los efectos más comunes son las curaciones, realizar daños o control de adversario, donde se engloban ralentizaciones o inmovilizaciones. Estos efectos tienen unos valores numéricos que constan de dos partes, un valor base y uno variable que depende de las estadísticas del campeón.

Explicado con un ejemplo, una habilidad de un campeón hace daño a un enemigo con un valor base de 150 puntos de vida y un valor variable que corresponde al 30\% daño de ataque del campeón. Si en un momento determinado el campeón tiene 300 de daño de ataque, el daño total de la habilidad se calcula como $150 + (0,3 * 300) = 240$.

En el caso de realizar daño a un rival, existen tres formas en las que puede realizar: daño físico, mágico y verdadero. Esto está determinado por la habilidad y rol del campeón.

\subsection{Estadísticas}
\label{estadisticas}
Las estadísticas son diferentes valores numéricos que determinan las capacidades de cada campeón en un área en concreto del juego. Estos valores se ven modificados por la compra de objetos (\ref{objetos}). A continuación se describen las estadísticas y que representan.
\begin{description}
	\item[Daño de ataque] Daño realizado con ataques básicos.
	\item[Probabilidad de crítico] Probabilidad de que un ataque básico haga el doble de daño.
	\item[Velocidad de ataque] Cantidad de ataques básicos que se pueden realizar por segundo.
	\item[Poder de habilidad] Modifica el daño que realizan las habilidades.
	\item[Velocidad de movimiento] Velocidad a la que un campeón se desplaza por el mapa.
	\item[Armadura] Cantidad en la que se ve reducida el daño físico que se recibe.
	\item[Resistencia mágica] Cantidad en la que se ve reducida el daño mágico que se recibe.
	\item[Penetración de armadura] Cantidad de la armadura del rival ignorada a la hora de realizar daño físico.
	\item[Penetración mágica] Cantidad de la resistencia mágica del rival ignorada a la hora de realizar daño mágico.
	\item[Vida] Daño que tiene que recibir un personaje para morir.
	\item[Maná] Coste de usar habilidades.
	\item[Robo de vida] Porcentaje de vida recuperado al dañar a un rival.
\end{description}

\subsection{Objetos}
\label{objetos}
Dentro de la base de cada equipo en el mapa está localizada la tienda. Aquí los jugadores pueden comprar objetos con el oro que han ido ganando con el progreso de la partida. Su función es modificar las estadísticas del campeón, para que las habilidades del mismo sean más eficaces contra los rivales. Los objetos que se comprar se quedan guardados en el inventario del campeón, el cual está limitado a seis objetos.

En el juego actual existen 222 objetos disponibles clasificados en cinco categorías.
\begin{description}
	\item[Iniciales] Objetos más relevantes al inicio de la partida.
	\item[Básicos] Objetos que mejoran una estadística.
	\item[Épicos] Objetos formados por la combinación de varios objetos básicos que mejoran varias estadísticas.
	\item[Legendarios] Objetos formados por la combinación de objetos épicos y/o básicos que mejoran varias estadísticas y proporcionan algún efecto adicional.
	\item[Míticos] Igual que los anteriores, pero limitado a uno en el inventario.
\end{description}

\subsection{Ligas}
Al igual que otros deportes, \textit{League of Legends} posee un sistema de ligas que clasifica a los jugadores que lo deseen dentro de su sistema de ligas, en base a la habilidad que demuestren en sus partidas. Ordenadas de menor a mayor habilidad, y con el porcentaje de pertenencia sobre el total \cite{misc:player-distribution}, las ligas del juego son las siguientes:
\begin{itemize}
	\tightlist
	\item Hierro - 2\%
	\item Bronce - 20\%
	\item Plata - 37\%
	\item Oro - 28\%
	\item Platino - 10\%
	\item Diamante - 1,5\%
	\item Maestro - 0,12\%
	\item Gran Maestro - 0,029\%
	\item Aspirante - 0,013\%
\end{itemize}
Desde Hierro a Diamante, cada una cuenta con cuatro divisiones (subcategorías dentro de cada liga), que van desde IV a I. Las tres restantes tiene una única división. Además, tanto Gran Maestro como Aspirante tienen plazas limitadas, 700 y 300 respectivamente.

La localización de cada jugador se basa en un sistema de puntos, ganándolos al ganar partidas y perdiéndolos al perder. En las ligas con divisiones, cuando el jugador alcanza 100 puntos pasa a la siguiente división, o el caso de estar en la divisón superior, subiría de liga. Por encima de Diamante no hay límite de puntos, y estando los jugadores ordenados por estos, la clasificación se realiza según el límite de plazas mencionado anteriormente.
 

%En aquellos proyectos que necesiten para su comprensión y desarrollo de unos conceptos teóricos de una determinada materia o de un determinado dominio de conocimiento, debe existir un apartado que sintetice dichos conceptos.
%
%Algunos conceptos teóricos de \LaTeX \footnote{Créditos a los proyectos de Álvaro López Cantero: Configurador de Presupuestos y Roberto Izquierdo Amo: PLQuiz}.
%
%\section{Secciones}
%
%Las secciones se incluyen con el comando section.
%
%\subsection{Subsecciones}
%
%Además de secciones tenemos subsecciones.
%
%\subsubsection{Subsubsecciones}
%
%Y subsecciones.
%
%
%\section{Referencias}
%
%Las referencias se incluyen en el texto usando cite \cite{wiki:latex}. Para citar webs, artículos o libros \cite{koza92}.
%
%
%\section{Imágenes}
%
%Se pueden incluir imágenes con los comandos standard de \LaTeX, pero esta plantilla dispone de comandos propios como por ejemplo el siguiente:
%
%\imagen{escudoInfor}{Autómata para una expresión vacía}
%
%
%
%\section{Listas de items}
%
%Existen tres posibilidades:
%
%\begin{itemize}
%	\item primer item.
%	\item segundo item.
%\end{itemize}
%
%\begin{enumerate}
%	\item primer item.
%	\item segundo item.
%\end{enumerate}
%
%\begin{description}
%	\item[Primer item] más información sobre el primer item.
%	\item[Segundo item] más información sobre el segundo item.
%\end{description}
%
%\begin{itemize}
%\item
%\end{itemize}
%
%\section{Tablas}
%
%Igualmente se pueden usar los comandos específicos de \LaTeX o bien usar alguno de los comandos de la plantilla.
%
%\tablaSmall{Herramientas y tecnologías utilizadas en cada parte del proyecto}{l c c c c}{herramientasportipodeuso}
%{ \multicolumn{1}{l}{Herramientas} & App AngularJS & API REST & BD & Memoria \\}{
%HTML5 & X & & &\\
%CSS3 & X & & &\\
%BOOTSTRAP & X & & &\\
%JavaScript & X & & &\\
%AngularJS & X & & &\\
%Bower & X & & &\\
%PHP & & X & &\\
%Karma + Jasmine & X & & &\\
%Slim framework & & X & &\\
%Idiorm & & X & &\\
%Composer & & X & &\\
%JSON & X & X & &\\
%PhpStorm & X & X & &\\
%MySQL & & & X &\\
%PhpMyAdmin & & & X &\\
%Git + BitBucket & X & X & X & X\\
%Mik\TeX{} & & & & X\\
%\TeX{}Maker & & & & X\\
%Astah & & & & X\\
%Balsamiq Mockups & X & & &\\
%VersionOne & X & X & X & X\\
%}

\capitulo{4}{Técnicas y herramientas}

\section{Técnicas}

\subsection{Scrum}
Scrum es una metodología de desarrollo ágil basada en los principios del manifiesto ágil\footnote{\url{http://agilemanifesto.org/}}. Esta metodología esta pensada para equipos por lo que ha sido adaptada para el desarrollo de este proyecto.

Scrum consiste en ciclos de trabajo iterativos denominados \textit{sprints}, con duración de una semana generalmente, en los que al terminar se entrega un producto funcional. Al final de cada \textit{sprint} se lleva a cabo una reunión con los tutores para comentar el desarrollo del \textit{sprint}, enseñar los avances y planear el siguiente ciclo de desarrollo.

\section{Herramientas}

\subsection{Git}
Git es un sistema de control de versiones distribuido. No se han considerado otras alternativas al ser un sistema ya conocido. Actualmente puede ser considerado el sistema con más usuarios y de más fama.

\subsection{GitHub}
GitHub es un servicio de alojamiento de repositorios de código basado en \textit{git}.

Esta plataforma se ha utilizado para alojar el proyecto, gestionar las tareas mediante \textit{issues} y planificar \textit{sprints} mediante \textit{milestones}.

\subsection{Jupyter}
\hrefFootnote{https://jupyter.org/}{Jupyter} es una plataforma, en formato de aplicación web, que permite desarrollar cuardernos que combinan texto y código. Desde ellos se puede ejecutar el código agrupado en celdas y ver la salida al mismo tiempo. Muy útil para realizar pruebas y mostrar resultados en tareas de análisis de datos.

\subsection{Django}
\hrefFootnote{https://www.djangoproject.com/}{Django} es un \textit{framework} para crear aplicaciones web escrito en \textit{Python} enfocado en el rápido desarrollo y facilitar gran parte del trabajo del desarrollo web.

\subsection{MongoDB}
\hrefFootnote{https://www.mongodb.com/}{MongoDB} es una base de datos NoSQL de tipo documental. Permite almacenar datos cuya estructura sea similar a JSON. En el ámbito de \textit{Big Data} es muy útil por su escalabilidad y velocidad. Adicionalmente proporciona herramientas para facilitar el proceso de transformación de los datos.

Se ha escogido esta plataforma porque da la posibilidad de guardar las respuestas en bruto de la API y, mediante agregaciones, poder realizar las transformaciones de datos necesarias de forma rápida y eficiente, además de poder cargarlos en la base de datos final.
\capitulo{5}{Aspectos relevantes del desarrollo del proyecto}

\section{Elección del proyecto}


\section{Explicación de la problemática}


\section{Desarrollo de \textit{notebooks} y web}


\section{Control de velocidad de peticiones por ratios de la API}


\section{Tolerancia y recuperación de fallos}


\section{Fases del proceso de recopilación y entrenamiento}

\subsection{Extracción jugadores}
\subsection{Extracción de partidas}
\subsection{Añadir posición}
\subsection{Generar transacciones}
\subsection{Generar conjuntos frecuentes}


\section{Elección del algoritmo}


\capitulo{6}{Trabajos relacionados}

\section{LoL Data Solution \cite{pino_2019}}
\textit{LoL Data Solution} es un proyecto desarrollado dentro del equipo MAD Lions. Su objetivo principal es la recopilación y agregación de partidas de jugadores del propio equipo y sus adversarios, para así sacar estadísticas sobre el rendimiento de los jugadores y usarlo para tomar mejores decisiones en el entrenamiento del equipo y en la preparación contra otros equipos de la competición.

\capitulo{7}{Conclusiones y Líneas de trabajo futuras}

\section{Conclusiones}
Habiendo terminado el desarrollo del proyecto, me siento satisfecho con el producto conseguido y considero que se han cumplido los objetivos propuestos al inicio. Como resultado se ha obtenido un proceso capaz de recopilar y analizar partidas, capaz de obtener información útil que mostrar de una forma intuitiva en una aplicación web.

He conseguido aplicar varias técnicas aprendidas durante el desarrollo de máster y me he dado cuenta de muchas consideraciones a tener en mente siempre que me encuentre en el desarrollo de proyectos que estén relacionados con obtención masiva de datos.

Para terminar, me siento contento con el trabajo que he realizado estos meses, ha sido una experiencia positiva y enriquecedora de la que estoy seguro voy a ser capaz de aplicar lo aprendido en mi futuro.

\section{Líneas de trabajo futuras}
En el estado actual del proyecto, la aplicación puede proporcionar una gran ayuda a nuevos jugadores, a la vez que puede permitir a los profesionales analizar el estado actual del juego. Pero hay varios puntos que pueden mejorarse para ofrecer una herramienta más completa, algunos de los cuales serían:
\begin{itemize}
	\item Tener en cuenta el orden de compra de cada objeto además del estado final.
	\item Se podrían usar algoritmos de \textit{clustering} para obtener grupos de objetos similares y proporcionar alternativas a los usuario sobre los conjuntos existentes.
	\item Hay objetos que a pesar de ser útiles, podrían no aparecer por ser muy dependientes de la situación, por ejemplo, un campeón enemigo concreto, se podría incorporar una sección con todos los objetos y mostrar situaciones concretas donde conviene comprarlos.
	\item A la hora de generar las transacciones no se tiene en cuenta la versión del juego en la que han sido jugadas las partidas, lo que puede derivar en que las transacciones mezclen partidas que se hayan jugado en distintas actualizaciones. En futuras versiones de la aplicación habría que controlarlo para asegurar la información más correcta.
	\item Actualmente solo se recopilan datos del servidor de Europa, una buena forma de ampliar la cantidad de datos sería expandir la recopilación al resto de servidores (América y Asia).
\end{itemize}


%\renewcommand\chaptername{Anexo}
%\renewcommand\thechapter{\Roman{chapter}}
%\setcounter{chapter}{0}

% Añadir entrada en el índice: Anexos
\appendix
\addcontentsline{toc}{part}{Apéndices}
\part*{Apéndices}

\apendice{Plan de Proyecto Software}

\section{Introducción}

\section{Planificación temporal}

\subsection{Sprint 0}

Sprint dedicado a tareas logísticas y preparación del inicio del proyecto.

\begin{itemize}
    \item Creación del repositorio de código.
    \item Solicitud de una clave permanente para la API a la desarrolladora del juego.
    \item Pruebas con la API para comprobar la existencia de los datos necesarios.
    \item Reunión para formalizar el inicio del desarrollo del proyecto.
\end{itemize}

\subsection{Sprint 1}

Primer sprint de desarrollo. Dedicado a la extracción de jugadores.

\begin{itemize}
    \item Documentación Sprint 0.
    \item Echar un ojo a wrappers existentes de la API para comprobar su viabilidad de uso.
    \item Desarrollo de un notebook para la extracción de jugadores, teniendo en cuenta límite de peticiones y tolerancia a fallos.
\end{itemize}

\subsection{Sprint 2}

\begin{itemize}
    \item Documentación Sprint 1.
    \item Modificar extracción de jugadores para añadir identificador de cuenta, necesario para el siguiente paso de recuperación de partidas.
    \item Desarrollo de un notebook para extraer las partidas de los jugadores recuperador previamente.
    \item Crear biblioteca con funciones comunes para realizar peticiones y guardar estados de ejecución.
\end{itemize}

\section{Estudio de viabilidad}

\subsection{Viabilidad económica}

\subsection{Viabilidad legal}



\apendice{Especificación de Requisitos}

\section{Introducción}

En este apéndice se describen los objetivos generales de la aplicación y se detallan sus requisitos, tanto funcionales como no funcionales.

\section{Objetivos generales}
\begin{itemize}
    \item Desarrollar un proceso ETL que sea capaz de recopilar los datos necesarios usando la API oficial de Riot Games\footnote{Riot Games es la desarrolladora de League of Legends}.
	\item Aplicar técnicas de aprendizaje no supervisado sobre los datos recopilados, en este caso algoritmos para la obtención de conjuntos frecuentes de objetos.
	\item Ser capaz de extraer conocimiento útil a partir de los datos obtenidos.
	\item Desarrollar una aplicación en la que se pueda consultar el conocimiento extraído.
	\item Que el producto final sea capaz de ayudar a los nuevos jugadores.
\end{itemize}

\section{Catálogo de requisitos}
\subsection{Requisitos funcionales}
\begin{itemize}
	\item \textbf{RF-1 Proceso ETL}: la aplicación debe ser capaz de recopilar, transformar y almacenar los datos para sus posterior uso.
	\begin{itemize}
		\item \textbf{RF-1.1 Extracción de jugadores}: la aplicación debe poder extraer los jugadores de las ligas más altas.
		\item \textbf{RF-1.2 Extracción de partidas}: la aplicación debe extraer las partidas de los jugadores obtenidos previamente.
		\item \textbf{RF-1.3 Generación de transacciones}: la aplicación debe transformar los datos de partidas en listados de objetos agrupados por campeón.
		\item \textbf{RF-1.4 Generación de conjuntos frecuentes}: la aplicación debe generar conjuntos frecuentes de objetos usando las transacciones obtenidas anteriormente.
	\end{itemize}
\end{itemize}

\begin{itemize}
	\item \textbf{RF-2 Consulta de información}: la aplicación debe ser capaz de recopilar, transformar y almacenar los datos para su posterior uso.
	\begin{itemize}
		\item \textbf{RF-2.1 Búsqueda por campeón}: el usuario debe poder buscar conjuntos frecuentes filtrando por un campeón concreto.
		\item \textbf{RF-2.2 Búsqueda en partida activa}: el usuario debe poder buscar conjuntos de objetos usando su nombre dentro del juego, de tal forma que se muestre la información adecuada para el campeón usado en ese momento.
	\end{itemize}
\end{itemize}

\subsection{Requisitos no funcionales}
\begin{itemize}
	\item \textbf{RNF-1 Usabilidad}: la aplicación debe ser intuitiva y fácil de usar.
	\item \textbf{RNF-2 Mantenibilidad}: debe ser sencillo añadir funcionalidad nueva a la aplicación.
	\item \textbf{RNF-3 Compatibilidad}: la aplicación debe poder funcionar en	los principales navegadores.
	\item \textbf{RNF-4 Responsividad}: la aplicación debe adaptarse al tamaño	de la pantalla.
\end{itemize}

\section{Especificación de requisitos}



\apendice{Especificación de diseño}

\section{Introducción}
En este anexo se explican los diseños que se han llevado a cabo para realizar los objetivos anteriores al igual que se describen los usados utilizados para conseguirlo.

\section{Diccionario de datos}

\subsection{Summoners}
\textit{Summoners} representa a la entidad del jugador.
\tablaSmall{Entidad del jugador}{lll}{jugador}{Variable & Descripción & Tipo \\}
{
accountId & Identificador de la cuenta a nivel de región & String \\
name & Nombre del jugador dentro del juego & String \\
region & Identificador de la región & String \\
puuid & Identificador global de la cuenta & String \\
summonerId & Identificador del jugador a nivel de región & String \\
}

\subsection{Matches}
\textit{Matches} es la entidad referida a las partidas jugadas. Se muestran solo los atributos relevantes para este problema ya el bruto contiene demasiada información que no aporta valor a este problema.
\tablaSmall{Entidad de la partida}{lll}{partida}{Variable & Descripción & Tipo \\}
{
	creation & Fecha y hora en la que comienza la partida & Timestamp \\
	region & Identificador de la región & String \\
	version & Identificador de la versión del juego en la que se ha jugado la partida & String \\
	gameId & Identificador de la partida & Integer \\
	teams & Lista que contiene a los dos equipos, siempre tiene tamaño dos & Array \\
	teams.participants & Lista que contiene a los jugadores del equipo, siempre tiene tamaño cinco & Array \\
	teams.participants.stats & Varias estadísticas del jugador al terminar la partida & Map \\
	teams.participants.stats.item0 & Identificador del primer objeto & Integer \\
	teams.participants.stats.item1 & Identificador del segundo objeto & Integer \\
	teams.participants.stats.item2 & Identificador del tercer objeto & Integer \\
	teams.participants.stats.item3 & Identificador del cuarto objeto & Integer \\
	teams.participants.stats.item4 & Identificador del quinto objeto & Integer \\
	teams.participants.stats.item5 & Identificador del sexto objeto & Integer \\
	teams.participants.championId & Identificador del campeón seleccionado por el jugador & Integer \\
	teams.participants.position & Posición en la que se ha jugador el campeón seleccionado por el jugador & String \\
}

\subsection{Transactions}
\textit{Transactions} es la entidad que representa los datos de entrada que se van a suministrar al algoritmo Apriori. Es el resultado de la fase de transformación de la \textit{pipeline}.
\tablaSmall{Transacciones}{lll}{transacciones}{Variable & Descripción & Tipo \\}
{
	champion & Identificador del jugador & Integer \\
	position & Posición en la que se ha jugado al campeón & String \\
	transactions & Listado de conjuntos de objetos que ha tenido el campeón en la posición & Array<Array> \\
	transactions.* & Conjunto de objetos con los que el campeón ha terminado la partida & Array<Integer> \\
}

\subsection{Frequent builds}
\textit{Frequent builds} son los datos que se muestran en la aplicación web al usuario final. Contienen la salida del algoritmo Apriori y atributos para identificar al campeón y posición.
\tablaSmall{Conjuntos frecuentes de objetos}{lll}{frequent-builds}{Variable & Descripción & Tipo \\}
{
	champion & Identificador del jugador & Integer \\
	games & Cantidad de partidas jugadas con el campeón llevado en una posición & Integer \\
	position & Posición en la que se ha jugado al campeón & String \\
	builds & Listado de conjuntos frecuentes de objetos & Array \\
	builds.support & Porcentaje de veces que aparece el conjunto de objetos en el listado de transacciones & Float \\
	builds.itemset & Conjunto de objetos frecuentes encontrado por el algoritmo & Array \\
}


\section{Diseño procedimental}
En esta sección se incluirán varios diagramas para comprender mejor cada una de las fases por las que se ha pasado hasta conseguir los conjuntos frecuentes que se van a mostrar al usuario.

\subsection{Extracción de jugadores}
La primera fase es la extracción de jugadores, se obtienen desde la API, se añade el identificador de cuenta y se guardan.
\begin{figure}[h]
	\centering
	\includegraphics[width=1\linewidth]{img/diag-summoners}
	\caption{Extracción de jugadores}
	\label{fig:diag-summoners}
\end{figure}

\subsection{Extracción de partidas}
A continuación se obtienen las partidas de cada jugador, se añade la posición y se guardan.
\begin{figure}[h]
	\centering
	\includegraphics[width=1\linewidth]{img/diag-matches}
	\caption{Extracción de partidas}
	\label{fig:diag-matches}
\end{figure}

\subsection{Generación de transacciones y conjuntos frecuentes}
A partir de las partidas, se ejecuta la agregación en MongoDB y se guardan. A continuación se usan las transacciones como entrada para el Apriori y se guarda la salida en la colección final.
\begin{figure}[h]
	\centering
	\includegraphics[width=1\linewidth]{img/diag-transform}
	\caption{Generación de transacciones y conjuntos frecuentes}
	\label{fig:diag-matches}
\end{figure}
\apendice{Documentación técnica de programación}

\section{Introducción}
En este apéndice se presenta todo lo que tiene que conocer un desarrollador para poder continuar con el desarrollo de la aplicación. Se describe la estructura de directorios del proyecto, cómo instalar la aplicación, etc.

El proyecto tiene dos partes diferenciadas en cuanto a desarrollo. La primera es una colección de \textit{notebooks} en la que se realizar pruebas con la API y algoritmos. La segunda es una aplicación web Django estándar.

\section{Estructura de directorios}

\dirtree{%
.1 / Directorio raíz.
.2 memoria/ - Documentación del proyecto.
	.3 img/ - Imágenes de la memoria.
	.3 tex/ - Secciones de la memoria.
	.3 memoria.tex - Código fuente de la memoria.
	.3 memoria.pdf - Memoria del proyecto.
	.3 bibliografia.bib - Fuentes bibliográficas.
.2 scripts and notebooks/ - Colección de notebooks y scripts.
	.3 outputs/ - Directorio donde se almacenan las salidas de los notebooks.
	.3 0\_extract\_players.ipynb - Notebook para extraer jugadores.
	.3 1\_extract\_matches.ipynb - Notebook para extraer partidas.
	.3 2\_download\_matches\_details.ipynb - Notebook para obtener detalles de cada partida.
	.3 3\_algorithms.ipynb - Notebook para probar algoritmos.
	.3 utils.py - Colección de funciones comunes para los notebooks.
	.3 api\_key.txt - Fichero que contiene la clave de la API.
	.3 requirements.txt - Fichero que lista las dependencias.
.2 web/ - Directorio del proyecto en Django.
	.3 betterbuilds/ - App de Django que contiene la web.
		.4 migrations/.
		.4 static/.
		.4 templates/.
		.4 \_\_init.py\_\_.
		.4 admin.py.
		.4 apps.py.
		.4 models.py.
		.4 tests.py.
		.4 urls.py.
		.4 views.py.
	.3 etl/ - App de Django que contiene el proceso ETL.
		.4 management/.
			.5 commands/.
		.4 migrations/.
		.4 \_\_init.py\_\_.
		.4 admin.py.
		.4 apps.py.
		.4 models.py.
		.4 tests.py.
		.4 views.py.
	.3 web/ - Directorio de configuración del proyecto Django.
		.4 \_\_init.py\_\_.
		.4 asgi.py.
		.4 settings.py.
		.4 urls.py.
		.4 utils.py.
		.4 wsgi.py.
	.3 .env - Fichero con variables de entorno.
	.3 manage.py - Script para interactuar con el proyecto.
	.3 requirements.txt - Fichero que lista las dependencias.
}

\section{Manual del programador}
\subsection{Notebooks}
Como los \textit{notebooks} han sido usados para realizar pruebas y probar la \textit{pipeline}, se recomienda probar en los mismos cualquier modificación a realizar en la recopilación y transformación de los datos.

También es recomendable usarlos para explorar los parámetros en las diversas peticiones a la API y sus respuestas.

\subsection{Aplicación web}
Al ser una aplicación estándar de \textit{Django}, se necesita tener conocimiento sobre el \textit{framework}. Para cualquier duda consultar en su \href{https://docs.djangoproject.com/en/3.2/}{documentación}\footnote{\url{https://docs.djangoproject.com/en/3.2/}}.

En este apartado se explican de forma básica los conceptos del \textit{framework} y cómo se han aplicado en este desarrollo.

Los proyectos escritos en \textit{Django} se organizan en los que llaman \textit{apps}. Cada \textit{app} representa cada módulo o servicio del proyecto, se componen de vistas y modelos propios e independientes del resto de \textit{apps}, aunque pueden integrarse para el correcto funcionamiento del proyecto. El producto desarrollado se compone de dos \textit{apps}, \textbf{betterbuilds} y \textbf{etl}.

La \textit{app} betterbuils contiene las vistas y modelos que se encargan de pedir y mostrar los datos necesarios en el navegador. Esto incluye la pantalla de inicio, el listado de campeones y la ficha del campeón. Sería la aplicación web comúnmente hablando.

Dentro de la \textit{app} etl se encuentran las distintas fases de la \textit{pipeline} codificadas como comandos de gestión del proyecto. Los comandos de gestión se ejecutan desde el terminal y funcionan como si se tratase de la ejecución de un \textit{script}. Para cada fase de la \textit{pipeline} existe un comando que la realiza de principio a fin. La relación entre los distintos comandos y la fase es la siguiente:
\begin{description}
	\item[Extracción de jugadores:] \code{import\_summoners.py}
	\item[Extracción de partidas:] \code{import\_matches.py}
	\item[Añadir posición:] \code{add\_position.py}
	\item[Generar transacciones:] \code{generate\_transactions.py}
	\item[Generar conjuntos frecuentes:] \code{import\_frequent\_builds.py}
\end{description}

\section{Instalación y ejecución del proyecto}
Dado que el proyecto se ha desarrollado usando el sistema operativo Ubuntu, las instrucciones de instalación están orientadas a ese sistema.

\subsection{Requisitos previos}
Antes de comenzar con la instalación del proyecto, el sistema tiene que tener instaladas las siguientes utilidades:
\begin{itemize}
	\item \hrefFootnote{https://docs.anaconda.com/anaconda/install/index.html}{Anaconda}
	\item \hrefFootnote{https://docs.mongodb.com/manual/administration/install-community/}{MongoDB}
	\item \hrefFootnote{https://git-scm.com/downloads}{Git}
\end{itemize}

También es necesario tener una clave de acceso a la API de Riot Games para poder hacer uso de la misma. Para ello hay que crearse una cuenta de desarrollo en \url{https://developer.riotgames.com/} y seguir las instrucciones de \url{https://developer.riotgames.com/docs/portal}.

\subsection{Instalación}\label{sec:instalacion}
La forma más cómoda de obtener el código del proyecto es mediante \textit{git}, para ello usar el siguiente comando:
\begin{lstlisting}
	$ git clone <url_del_repositorio>
\end{lstlisting}
Siendo \url{https://github.com/IvanBeke/TFM} la URL del repositorio.

Ya con el proyecto descargado, el siguiente paso es instalar las dependencias. Para ello hay que ejecutar el comando:
\begin{lstlisting}
	$ pip install -r scripts and notebooks/requirementes.txt
\end{lstlisting}
Para la aplicación se recomienda usar un entorno virtual para tener mejor control sobre las dependencias exactas. Para crearlo y activarlo:
\begin{lstlisting}
	$ conda create --name django
	$ conda activate django
\end{lstlisting}
Una vez está el entorno en uso se instalan las dependencias con el comando:
\begin{lstlisting}
	$ pip install -r web/requirementes.txt
\end{lstlisting}

\subsubsection{Variables de entorno}
Para el uso de \textit{notebooks} se necesita crear un fichero \code{api\_key.txt}, en el directorio \code{scripts and notebooks} que contiene la clave de acceso a la API.

Para la aplicación web, hay que crear un fichero \code{.env}, en el directorio \textit{web} que contenga las variables de entorno. También se pueden definir en el propio sistema. El listado de variables a las que hay que dar un valor es el siguiente:
\begin{itemize}
	\item \code{RIOT\_API\_KEY}: clave de acceso a la API.
	\item \code{RIOT\_API\_REGION}: región sobre la que se van a ejecutar las peticiones.
	\item \code{SECRET\_KEY}: clave que usa \textit{Django} para seguridad interna.
	\item \code{DEBUG}: define si está activado el modo depuración de la aplicación.
\end{itemize}

\subsection{Ejecución}
Partiendo del directorio raíz se muestra como ejecutar los \textit{notebooks} o la aplicación web.
\subsubsection{Notebooks}
\begin{lstlisting}
	$ cd scripts and notebooks
	$ jupyter notebook
\end{lstlisting}
\subsubsection{Aplicación web}
\begin{lstlisting}
	$ cd web
	$ python manage.py runserver
\end{lstlisting}
Con eso se lanzaría la aplicación web. Para la ejecución de las distintas fases del proceso de recopilación y entrenamiento se ejecutarían los siguientes comandos:
\begin{lstlisting}
	$ cd web
	$ python manage.py import_summoners
	$ python manage.py import_matches
	$ python manage.py add_position
	$ python manage.py generate_transaction
	$ python manage.py generate_frequent_builds
\end{lstlisting}

%\section{Pruebas del sistema}

\apendice{Documentación de usuario}

\section{Introducción}
En este apéndice se explican los requisitos que debe cumplir el usuario para ejecutar la aplicación, cómo lanzarla y cómo usarla.

\section{Requisitos de usuarios}
Al tratarse de una aplicación web, los requisitos que debe cumplir el usuario son los siguientes:

\begin{itemize}
	\tightlist
	\item Navegador web instalado, la aplicación se ha probado en \textit{Firefox} y \textit{Chrome}.
	\item JavaScript activo en el navegador.
\end{itemize}


\section{Instalación}
Debido a que se proporciona una aplicación web, no es necesario instalarla para poder usarla. Sin embargo, si se quiere proceder a la instalación, se pueden seguir las instrucciones en la sección \ref{sec:instalacion}.


\section{Manual del usuario}
En esta sección se enseña al usuario como manejar la aplicación.

\subsection{Inicio}
Nada más entrar a la aplicación se muestra la pantalla de inicio (figura~\ref{fig:inicio}). En ella se puede ver un texto de presentación de la aplicación web y se da acceso al resto de funcionalidades.

\begin{figure}[h]
	\centering
	\includegraphics[width=1\linewidth]{img/0.inicio}
	\caption{Pantalla de inicio}
	\label{fig:inicio}
\end{figure}

Desde aquí se puede navegar al listado de campeones (sección~\ref{sec:listado}) y se puede buscar una partida en curso (sección~\ref{sec:partida}). Al estar localizados los enlaces en la cabecera, también se puede acceder desde el resto de la aplicación.

\subsection{Listado de campeones}\label{sec:listado}
En esta pantalla se presenta un listado con todos los campeones actuales del juego (figura~\ref{fig:listado}). También se encuentra un buscador con el que se puede filtrar el listado (figura~\ref{fig:buscador}), para encontrar de forma más rápida a un campeón concreto.

\begin{figure}[h]
	\centering
	\includegraphics[width=1\linewidth]{img/1.listado}
	\caption{Listado de campeones}
	\label{fig:listado}
\end{figure}
\begin{figure}[h]
	\centering
	\includegraphics[width=1\linewidth]{img/2.buscador}
	\caption{Buscador}
	\label{fig:buscador}
\end{figure}

Una vez localizado el campeón se puede pinchar en él para navegar hacia su ficha (sección~\ref{sec:ficha}) y consultar los conjuntos de objetos.

\subsection{Ficha del campeón}\label{sec:ficha}
La ficha del campeón (figura~\ref{fig:ficha} y \ref{fig:ficha2}) se divide en una sección de presentación y en otra con los conjuntos de objetos. En la parte de presentación se muestra la foto del campeón, el número de partidas analizadas, un texto sobre la historia del mismo y consejos para su manejo en partida.

\begin{figure}[h]
	\centering
	\includegraphics[width=1\linewidth]{img/3.campeon}
	\caption{Ficha del campeón}
	\label{fig:ficha}
\end{figure}
\begin{figure}[h]
	\centering
	\includegraphics[width=1\linewidth]{img/4.campeon}
	\caption{Ficha de otro campeón}
	\label{fig:ficha2}
\end{figure}

La parte de los objetos está dividida en pestañas, una para cada posición y una extra para las partidas en las que no ha sido posible identificarla. Se encuentran ordenadas por el porcentaje de partidas jugadas en esa posición.

Dentro de cada pestaña la estructura es similar. Se presenta un listado de con los conjuntos de objetos que se han obtenido del análisis de partidas, junto con el porcentaje de elección de esa combinación, representado por una barra de progreso para una mejor usabilidad. Los conjuntos se encuentran ordenados por este porcentaje.

\subsection{Partida actual}\label{sec:partida}
Para los jugadores que estén en partida y quieran realizar una consulta sobre el campeón que están jugando actualmente se da la opción de poder buscarse a si mismos mediante el apodo dentro del juego. Esta búsqueda lleva directamente al jugador a la ficha del personaje con el que se encuentra jugando (figura ~\ref{fig:partida}).

\begin{figure}[h]
	\centering
	\includegraphics[width=1\linewidth]{img/5.partida}
	\caption{Búsqueda por jugador}
	\label{fig:partida}
\end{figure}

En el caso de que el jugador no se encuentre en partida para el jugador buscado (figura~\ref{fig:no-partida}) o no exista el jugador (figura~\ref{fig:no-existes}), se muestra un mensaje de error avisando al usuario.

\begin{figure}[h]
	\centering
	\includegraphics[width=1\linewidth]{img/6.no-partida}
	\caption{El jugador no está en partida}
	\label{fig:no-partida}
\end{figure}
\begin{figure}
	\centering
	\includegraphics[width=1\linewidth]{img/7.no-existes}
	\caption{No existe el jugador con el nombre introducido}
	\label{fig:no-existes}
\end{figure}


\bibliographystyle{plain}
\bibliography{bibliografia}

\end{document}
